\documentclass[1pt]{report}
\usepackage{listings}
\usepackage{color}
%New colors defined below
\definecolor{codegreen}{rgb}{0,0.6,0}
\definecolor{codegray}{rgb}{0.5,0.5,0.5}
\definecolor{codepurple}{rgb}{0.58,0,0.82}
\definecolor{backcolour}{rgb}{0.95,0.95,0.92}

%Code listing style named "mystyle"
\lstdefinestyle{mystyle}{
  backgroundcolor=\color{backcolour},   commentstyle=\color{codegreen},
  keywordstyle=\color{magenta},
  basicstyle=\tiny\texttt,
  numberstyle=\tiny\color{codegray},
  stringstyle=\color{codepurple},
  basicstyle=\footnotesize,
  breakatwhitespace=false,         
  breaklines=true,                 
  captionpos=b,                    
  keepspaces=true,                 
  numbers=left,                    
  numbersep=5pt,                  
  showspaces=false,                
  showstringspaces=false,
  showtabs=false,                  
  tabsize=2
}

%"mystyle" code listing set
\lstset{style=mystyle}

\usepackage{hyperref}
\hypersetup{
    colorlinks=true,
    linkcolor=blue,
    filecolor=magenta,      
    urlcolor=cyan,
}
 
\urlstyle{same}

\usepackage{geometry}
 \geometry{
 a4paper,
 total={210mm,297mm},
 left=30mm,
 right=30mm,
 top=30mm,
 bottom=30mm,
 }

\title{Algorithm Code Book}
\author{Tanvir Hasan Anick}


\begin{document}
\maketitle
\tableofcontents
\newpage

%%Begin Data Structure
\chapter{Data Structure}
\section{Trie}
\subsection{Static Trie}
\lstinputlisting[language=C++]{../Algorithm/Trie/StaticTrie.cpp}
\section{RMQ}
\subsection{Bit}
\subsubsection{1D Bit}
\lstinputlisting[language=C++]{../Algorithm/Bit/Bit(1D).cpp}
\subsubsection{2D Bit}
\lstinputlisting[language=C++]{../Algorithm/Bit/Bit(2D).cpp}
\subsection{Square Root Decompostion}
\lstinputlisting[language=C++]{../Algorithm/Square_Root_Decomposition/squareRootDecomposition.cpp}
\subsection{MO's Algorithm}
\lstinputlisting[language=C++]{../Algorithm/MO's_Algorithm/MOs_Algorithm.cpp}
\subsection{Segment Tree}
\subsubsection{Lazy Propagration1}
\lstinputlisting[language=C++]{../Algorithm/RMQ-Segment_Tree/Segment_Tree_Lazy1.cpp}
\subsubsection{Lazy Propagration2}
\lstinputlisting[language=C++]{../Algorithm/RMQ-Segment_Tree/Segment_Tree_Lazy2.cpp}
\subsubsection{Segment Tree Variant 1}
\lstinputlisting[language=C++]{../Algorithm/RMQ-Segment_Tree/Segment_Tree_Variant1(CumulativeFrequency).cpp}
\subsubsection{Segment Tree Variant 2}
\lstinputlisting[language=C++]{../Algorithm/RMQ-Segment_Tree/Segment_Tree_Variant2(Maximum_sum_in_range).cpp}
\subsubsection{Segment Tree Variant 3}
\lstinputlisting[language=C++]{../Algorithm/RMQ-Segment_Tree/Segment_Tree_Variant3(Correct_maximum_bracket_seq).cpp}
\subsection{Sliding Window RMQ}
\lstinputlisting[language=C++]{../Algorithm/RMQ-Sliding_Window/Sliding_Window_Rmq.cpp}
\subsection{Sparse Table}
\lstinputlisting[language=C++]{../Algorithm/RMQ-Sparse_Table/Sparse_Table.cpp}

\section{Heavy Light Decomposition}
\lstinputlisting[language=C++]{../Algorithm/HLD/Heavy_LIght_Decomposition.cpp}

\section{Ternary Bit Mask}
\lstinputlisting[language=C++]{../Algorithm/Ternary_Bit_Mask/Tarnary_Bit_Mask.cpp}

%%End Data Structure



%%Begin Graph Theory
\chapter{Graph Theory}
\section{DFS}
\subsection{Bicoloring}
\lstinputlisting[language=C++]{../Algorithm/DFS/Bicoloring.cpp}
\subsection{Cycle Finding}
\lstinputlisting[language=C++]{../Algorithm/DFS/Cycle_Finding.cpp}
\section{Topological Sort}
\lstinputlisting[language=C++]{../Algorithm/Topological_Sort/Top_Sort_With_Dfs.cpp}
\section{Havel Hakimi}
\lstinputlisting[language=C++]{../Algorithm/Havel_Hakimi/HavelHakimi.cpp}
\section{Articulation Point/Bridge}
\subsection{Find Articulation Point:}
\lstinputlisting[language=C++]{../Algorithm/Articulation_point/FindArticulationPoint.cpp}
\subsection{Find Bridge version 1:}
\lstinputlisting[language=C++]{../Algorithm/Articulation_point/ArticulationPointFindBridge1.cpp}
\subsection{Find Bridge version 2:}
\lstinputlisting[language=C++]{../Algorithm/Articulation_point/ArticulationPointFindBridge2.cpp}

%%End Graph Theory

%%Begin Flow networks/ matching
\chapter{Flow networks/ matching}
\section{Max Flow}
\lstinputlisting[language=C++]{../Algorithm/Max_Flow/Max_Flow.cpp}

%%End Flow network/matching

%%Begin Dynamic programming
\chapter{Dynamic programming}
\section{Edit Distance}
\lstinputlisting[language=C++]{../Algorithm/Edit_Distance/Edit_Destance.cpp}

%%End Dynamic programming

%%Begin String algorithm
\chapter{Strings}
\section{KMP}
\href{https://tanvir002700.wordpress.com/2015/03/03/kmp-knuth-morris-pratt-algorithm/}{Tutorial}
\lstinputlisting[language=C++]{../Algorithm/KMP/kmp.cpp}
\section{Aho Corasick}
\subsection{Aho Corasick with Dynamic Trie}
\lstinputlisting[language=C++]{../Algorithm/Aho_Corasick/Aho-corasick(DynamicTrie).cpp}
\subsection{Aho Corasick with Static Trie}
\lstinputlisting[language=C++]{../Algorithm/Aho_Corasick/Aho-corasick(staticTrie).cpp}
\section{Manacher's Algorithm}
\lstinputlisting[language=C++]{../Algorithm/Manacher's_Algorithm/Manachers.cpp}

%%End String Algorithm


%%Begin Computational Geometry
\chapter{Computational geometry}

%%End Computational Geometry

%%Begin Math
\chapter{Math}
\section{Reduce Ratio}
$\left(\frac{A}{B}\right)$ ratio reduce to $\left(\frac{x}{y}\right)$
\lstinputlisting[language=C++]{../Algorithm/Number_Theory/A,B_reduce_to_x,y_ratio.cpp}
\section{Floyd's Cycle Finding algorithm}
\lstinputlisting[language=C++]{../Algorithm/Floyd's_Cycle_Finding_algorithm/Floyd'sCycleFindingAlgorithm.cpp}

%%End Math

%%Begin Number Theory
\chapter{Number Theory}
\section{NCR}
\subsection{Lucas Theorem}
\lstinputlisting[language=C++]{../Algorithm/Number_Theory/NCR(LucasTheorem).cpp}

%%End Number Theory

\end{document}